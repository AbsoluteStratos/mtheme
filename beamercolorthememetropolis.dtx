% \iffalse meta-comment -------------------------------------------------------
% Copyright 2015 Matthias Vogelgesang and the LaTeX community. A full list of
% contributors can be found at
%
%     https://github.com/matze/mtheme/graphs/contributors
%
% and the original template was based on the HSRM theme by Benjamin Weiss.
%
% This work is licensed under a Creative Commons Attribution-ShareAlike 4.0
% International License (https://creativecommons.org/licenses/by-sa/4.0/).
% ------------------------------------------------------------------------- \fi
% \iffalse
%<*driver>
\ProvidesFile{beamercolorthememetropolis.dtx}
%</driver>
%<package>\NeedsTeXFormat{LaTeX2e}
%<package>\ProvidesPackage{beamercolorthememetropolis}
%<*package>
    [2015/06/12 1.0.0 A Modern Beamer Color Theme]
%</package>
%
%<*driver>
\documentclass{ltxdoc}
\usepackage{beamercolorthememetropolis}
\EnableCrossrefs
\CodelineIndex
\RecordChanges
\begin{document}
  \DocInput{beamercolorthememetropolis.dtx}
  \PrintChanges
  \PrintIndex
\end{document}
%</driver>
% \fi
%
% \CheckSum{0}
%
% \CharacterTable
%  {Upper-case    \A\B\C\D\E\F\G\H\I\J\K\L\M\N\O\P\Q\R\S\T\U\V\W\X\Y\Z
%   Lower-case    \a\b\c\d\e\f\g\h\i\j\k\l\m\n\o\p\q\r\s\t\u\v\w\x\y\z
%   Digits        \0\1\2\3\4\5\6\7\8\9
%   Exclamation   \!     Double quote  \"     Hash (number) \#
%   Dollar        \$     Percent       \%     Ampersand     \&
%   Acute accent  \'     Left paren    \(     Right paren   \)
%   Asterisk      \*     Plus          \+     Comma         \,
%   Minus         \-     Point         \.     Solidus       \/
%   Colon         \:     Semicolon     \;     Less than     \<
%   Equals        \=     Greater than  \>     Question mark \?
%   Commercial at \@     Left bracket  \[     Backslash     \\
%   Right bracket \]     Circumflex    \^     Underscore    \_
%   Grave accent  \`     Left brace    \{     Vertical bar  \|
%   Right brace   \}     Tilde         \~}
%
%
% \changes{1.0.0}{2015/06/12}{Initial Stable Release.}
%
% \DoNotIndex{\newcommand,\newenvironment}
%

%
% \StopEventually{}
%
% \subsection{Color Theme}
%
% \iffalse
%<*package>
% \fi
%
% Options
%
%    \begin{macrocode}
\newif\if@beamer@metropolis@blockbg
\@beamer@metropolis@blockbgfalse
\DeclareOptionBeamer{blockbg}{
    \@beamer@metropolis@blockbgtrue
}
%    \end{macrocode}
%
% darkcolors
%
%    \begin{macrocode}
\newif\if@beamer@metropolis@darkcolors
\@beamer@metropolis@darkcolorsfalse
\DeclareOptionBeamer{darkcolors}{
    \@beamer@metropolis@darkcolorstrue
}
%    \end{macrocode}
%
% Unknown option error handling
%
%    \begin{macrocode}
\DeclareOptionBeamer*{%
  \PackageWarning{beamercolorthememetropolis}{Unknown option `\CurrentOption'}%
}
\ProcessOptionsBeamer
%    \end{macrocode}
%
% Colors
%
%    \begin{macrocode}
\definecolor{mDarkBrown}{HTML}{604c38}
\definecolor{mDarkTeal}{HTML}{23373b}
\definecolor{mLightBrown}{HTML}{EB811B}
\definecolor{mLightGreen}{HTML}{14B03D}
%    \end{macrocode}
%
% Base Colors
%
%    \begin{macrocode}
\if@beamer@metropolis@darkcolors
  \setbeamercolor{normal text}{%
    fg=black!2,
    bg=mDarkTeal
  }
\else
  \setbeamercolor{normal text}{%
    fg=mDarkTeal,
    bg=black!2
  }
\fi
\setbeamercolor{alerted text}{%
  fg=mLightBrown
}
\setbeamercolor{example text}{%
  fg=mLightGreen
}
%    \end{macrocode}
%
% Derived Colors
%
%    \begin{macrocode}
\setbeamercolor{titlelike}{use=normal text, parent=normal text}
\setbeamercolor{structure}{%
  fg=normal text.fg
}
%    \end{macrocode}
%
% Frame titles and plain slides
%
%    \begin{macrocode}
\setbeamercolor{frametitle}{use=palette primary, parent=palette primary}
%    \end{macrocode}
%
% The “primary” palette should be used for the most important navigational
% elements, and possibly of other elements.
% The metropolis color theme uses it for frame titles and slides.
%
%    \begin{macrocode}
\setbeamercolor{palette primary}{%
  use=normal text,
  fg=normal text.bg,
  bg=normal text.fg
}
%    \end{macrocode}
%
% Progress bar and title separator
%
%    \begin{macrocode}
\setbeamercolor{title separator}{use=progress bar, parent=progress bar}
\setbeamercolor{progress bar}{%
  use=alerted text,
  fg=alerted text.fg,
  bg=normal text.bg!50!fg
}
%    \end{macrocode}
%
% Blocks
%
%    \begin{macrocode}
\if@beamer@metropolis@blockbg
  \setbeamercolor{block title}{%
    use=normal text,
    fg=normal text.fg,
    bg=normal text.bg!80!fg
  }
\else
  \setbeamercolor{block title}{use=normal text, parent=normal text}
\fi
\setbeamercolor{block title alerted}{%
    use={block title, alerted text},
    bg=block title.bg,
    fg=alerted text.fg
}
\setbeamercolor{block title example}{%
    use={block title, example text},
    bg=block title.bg,
    fg=example text.fg
}
\setbeamercolor{block body alerted}{use=block body, parent=block body}
\setbeamercolor{block body example}{use=block body, parent=block body}
\setbeamercolor{block body}{
  use={block title, normal text},
  bg=block title.bg!50!normal text.bg
}
%    \end{macrocode}
%
% Footnotes
%
%    \begin{macrocode}
\setbeamercolor{footnote}{fg=normal text.fg!90}
\setbeamercolor{footnote mark}{fg=.}
\mode<all>
%    \end{macrocode}

%
% \iffalse
%</package>
% \fi
%
% \Finale
\endinput
