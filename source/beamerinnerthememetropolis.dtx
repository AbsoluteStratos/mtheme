% \iffalse meta-comment -------------------------------------------------------
% Copyright 2015 Matthias Vogelgesang and the LaTeX community. A full list of
% contributors can be found at
%
%     https://github.com/matze/mtheme/graphs/contributors
%
% and the original template was based on the HSRM theme by Benjamin Weiss.
%
% This work is licensed under a Creative Commons Attribution-ShareAlike 4.0
% International License (https://creativecommons.org/licenses/by-sa/4.0/).
% ------------------------------------------------------------------------- \fi
% \iffalse
%<driver> \ProvidesFile{beamerinnerthememetropolis.dtx}
%<*package>
\NeedsTeXFormat{LaTeX2e}
\ProvidesPackage{beamerinnerthememetropolis}[2015/12/04 Metropolis inner theme]
%</package>
%<driver> \documentclass{ltxdoc}
%<driver> \usepackage{beamerinnerthememetropolis}
%<driver> \begin{document}
%<driver> \DocInput{beamerinnerthememetropolis.dtx}
%<driver> \end{document}
% \fi
% \CheckSum{0}
% \StopEventually{}
% \iffalse
%<*package>
% ------------------------------------------------------------------------- \fi
%
% \subsection{\themename inner theme}
%
% A |beamer| inner theme dictates the style of the frame elements traditionally
% set in the ``body'' of each slide. These include:
%
% \begin{itemize}
%   \item title, part, and section pages;
%   \item itemize, enumerate, and description environments;
%   \item block environments including theorems and proofs;
%   \item figures and tables; and
%   \item footnotes and plain text.
% \end{itemize}
%
% Load required packages.
%    \begin{macrocode}
\RequirePackage{etoolbox}
\RequirePackage{calc}
\RequirePackage{pgfopts}
\RequirePackage{tikz}
%    \end{macrocode}
%
%
%
% \subsubsection{Options}
%
% \begin{macro}{block}
% This option controls the block style.
%    \begin{macrocode}
\pgfkeys{
  /metropolis/inner/block/.cd,
    .is choice,
    transparent/.code=\setlength{\@metropolis@blockskip}{0ex},
    fill/.code=\setlength{\@metropolis@blockskip}{1ex},
}
%    \end{macrocode}
% \end{macro}
%
% \begin{macro}{titleformat title}
% Control the titleformat of the title
%    \begin{macrocode}
\pgfkeys{
  /metropolis/inner/titleformat title/.cd,
    .is choice,
    regular/.code={%
      \let\@metropolis@titleformat\@empty%
      \setbeamerfont{title}{shape=\normalfont}%
    },
    smallcaps/.code={%
      \let\@metropolis@titleformat\@empty%
      \setbeamerfont{title}{shape=\scshape}%
    },
    allsmallcaps/.code={%
      \let\@metropolis@titleformat\MakeLowercase%
      \setbeamerfont{title}{shape=\scshape}%
      \PackageWarning{beamerthememetropolis}{%
        Be aware that titleformat title=allsmallcaps can lead to problems%
      }
    },
    allcaps/.code={%
      \let\@metropolis@titleformat\MakeUppercase%
      \setbeamerfont{title}{shape=\normalfont}
      \PackageWarning{beamerthememetropolis}{%
        Be aware that titleformat title=allcaps can lead to problems%
      }
    },
}
%    \end{macrocode}
% \end{macro}
%
% \begin{macro}{titleformat subtitle}
% Control the titleformat of the subtitle
%    \begin{macrocode}
\pgfkeys{
  /metropolis/inner/titleformat subtitle/.cd,
    .is choice,
    regular/.code={%
      \let\@metropolis@subtitleformat\@empty%
      \setbeamerfont{subtitle}{shape=\normalfont}%
    },
    smallcaps/.code={%
      \let\@metropolis@subtitleformat\@empty%
      \setbeamerfont{subtitle}{shape=\scshape}%
    },
    allsmallcaps/.code={%
      \let\@metropolis@subtitleformat\MakeLowercase%
      \setbeamerfont{subtitle}{shape=\scshape}%
      \PackageWarning{beamerthememetropolis}{%
        Be aware that titleformat subtitle=allsmallcaps can lead to problems%
      }
    },
    allcaps/.code={%
      \let\@metropolis@subtitleformat\MakeUppercase%
      \setbeamerfont{subtitle}{shape=\normalfont}%
      \PackageWarning{beamerthememetropolis}{%
        Be aware that titleformat subtitle=allcaps can lead to problems%
      }
    },
}
%    \end{macrocode}
% \end{macro}
%
% \begin{macro}{titleformat section}
% Control the titleformat of the section title
%    \begin{macrocode}
\pgfkeys{
  /metropolis/inner/titleformat section/.cd,
    .is choice,
    regular/.code={%
      \let\@metropolis@sectiontitleformat\@empty%
      \setbeamerfont{section title}{shape=\normalfont}%
    },
    smallcaps/.code={%
      \let\@metropolis@sectiontitleformat\@empty%
      \setbeamerfont{section title}{shape=\scshape}%
    },
    allsmallcaps/.code={%
      \let\@metropolis@sectiontitleformat\MakeLowercase%
      \setbeamerfont{section title}{shape=\scshape}%
      \PackageWarning{beamerthememetropolis}{%
        Be aware that titleformat section=allsmallcaps can lead to problems%
      }
    },
    allcaps/.code={%
      \let\@metropolis@sectiontitleformat\MakeUppercase%
      \setbeamerfont{section title}{shape=\normalfont}%
      \PackageWarning{beamerthememetropolis}{%
        Be aware that titleformat section=allcaps can lead to problems%
      }
    },
}
%    \end{macrocode}
% \end{macro}
%
% \begin{macro}{sectionpage}
% The |sectionpage| option defines the behaviour of the sectionpage.
%    \begin{macrocode}
\pgfkeys{
  /metropolis/inner/sectionpage/.cd,
    .is choice,
    none/.code=\@metropolis@sectionpage@none,
    simple/.code=\@metropolis@sectionpage@simple,
    progressbar/.code=\@metropolis@sectionpage@progressbar,
}
%    \end{macrocode}
% \end{macro}
%
% \begin{macro}{\@metropolis@inner@setdefaults}
% Set default values for inner theme options.
%    \begin{macrocode}
\newcommand{\@metropolis@inner@setdefaults}{
  \pgfkeys{/metropolis/inner/.cd,
    sectionpage=progressbar,
    block=transparent,
    titleformat title=regular,
    titleformat subtitle=regular,
    titleformat section=regular,
  }
}
%    \end{macrocode}
% \end{macro}
%
%
%
% \subsubsection{Title page}
%
% \begin{macro}{\@metropolis@titleformat}
% Define hooks to change the case format of the titles.
%    \begin{macrocode}
\def\@metropolis@titleformat#1{#1}
\def\@metropolis@subtitleformat#1{#1}
\def\@metropolis@sectiontitleformat#1{#1}
%    \end{macrocode}
% \end{macro}
%
% To make the |\MakeLowercase| and |\MakeUppercase| macros work in the
% sectiontitle we have to patch |\sectionentry| and |\beamer@section|. This
% solution was suggested by Enrico Gregorio in an answer to
% \href{http://tex.stackexchange.com/questions/112526/}{this StackExchange
% question}.
%
%    \begin{macrocode}
\patchcmd{\sectionentry}
  {\def\insertsectionhead{#2}}
  {\def\insertsectionhead{\@metropolis@sectiontitleformat{#2}}}
  {}
  {\PackageError{beamerinnerthememetropolis}{Patching section title failed}}
\patchcmd{\beamer@section}
  {\def\insertsectionhead{\hyperlink{Navigation\the\c@page}{#1}}}
  {\def\insertsectionhead{\hyperlink{Navigation\the\c@page}{%
  \@metropolis@sectiontitleformat{#1}}}}
  {}
  {\PackageError{beamerinnerthememetropolis}{Patching section title failed}}
%    \end{macrocode}
%
% \begin{macro}{title page}
% Template for the title page. Each element is only typset if it is defined
% by the user. If |\subtitle| is empty, for example, it won't leave a blank
% space on the title slide.
%    \begin{macrocode}
\setbeamertemplate{title page}{
  \begin{minipage}[b][\paperheight]{\textwidth}
    \ifx\inserttitlegraphic\@empty\else\usebeamertemplate*{title graphic}\fi
    \vfill%
    \ifx\inserttitle\@empty\else\usebeamertemplate*{title}\fi
    \ifx\insertsubtitle\@empty\else\usebeamertemplate*{subtitle}\fi
    \usebeamertemplate*{title separator}
%    \end{macrocode}
%
% Beamer's definition of |\insertauthor| is always nonempty, so we have
% to test another macro initialized by |\author{...}| to see if the user has
% defined an author. This solution was suggested by Enrico Gregorio in an
% answer to \href{https://tex.stackexchange.com/questions/241306/}{this
% Stack Exchange question}.
%
%    \begin{macrocode}
    \ifx\beamer@shortauthor\@empty\else\usebeamertemplate*{author}\fi
    \ifx\insertdate\@empty\else\usebeamertemplate*{date}\fi
    \ifx\insertinstitute\@empty\else\usebeamertemplate*{institute}\fi
    \vfill
    \vspace*{1mm}
  \end{minipage}
}
%    \end{macrocode}
% \end{macro}%
%
% Normal people should use |\maketitle| or |\titlepage| instead of using the
% |title page| beamer template directly. Beamer already defines these macros,
% but we patch them here to make the title page |[plain]| by default, remove
% |\@thanks|, and ensure the title frame number doesn't count.
%
% \begin{macro}{\maketitle}
% \begin{macro}{\titlepage}
%
%   Inserts the title frame, or causes the current frame to use the
%   |title page| template.
%
%    \begin{macrocode}
\def\maketitle{%
  \ifbeamer@inframe
    \titlepage
  \else
    \frame[plain,noframenumbering]{\titlepage}
  \fi
}
\def\titlepage{%
  \usebeamertemplate{title page}
}
%    \end{macrocode}
% \end{macro}
% \end{macro}
%
% \begin{macro}{title graphic}
%   Set the title graphic in a zero-height box, so it doesn't change the
%   position of other elements.
%    \begin{macrocode}
\setbeamertemplate{title graphic}{
  \vbox to 0pt {
    \vspace*{2em}
    \inserttitlegraphic%
  }%
  \nointerlineskip%
}
%    \end{macrocode}
% \end{macro}
%
% \begin{macro}{title}
%   Set the title on the title page.
%    \begin{macrocode}
\setbeamertemplate{title}{
  \raggedright%
  \linespread{1.0}%
  \@metropolis@titleformat{\inserttitle}%
  \par%
  \vspace*{0.5em}
}
%    \end{macrocode}
% \end{macro}
%
% \begin{macro}{subtitle}
%   Set the subtitle on the title page.
%    \begin{macrocode}
\setbeamertemplate{subtitle}{
  \@metropolis@subtitleformat{\insertsubtitle}%
  \par%
  \vspace*{0.5em}
}
%    \end{macrocode}
% \end{macro}
%
% \begin{macro}{title separator}
%   Template to set the title graphic in a zero-height box. (It won't
%   change the position of other elements.)
%    \begin{macrocode}
\setbeamertemplate{title separator}{
  \begin{tikzpicture}
    \draw[fg, fill=fg] (0,0) rectangle (\textwidth, 0.4pt);
  \end{tikzpicture}%
  \par%
}
%    \end{macrocode}
% \end{macro}
%
% \begin{macro}{author}
%   Set the author on the title page.
%    \begin{macrocode}
\setbeamertemplate{author}{
  \vspace*{2em}
  \insertauthor%
  \par%
  \vspace*{0.25em}
}
%    \end{macrocode}
% \end{macro}
%
% \begin{macro}{date}
%   Set the date on the title page.
%    \begin{macrocode}
\setbeamertemplate{date}{
  \insertdate%
  \par%
}
%    \end{macrocode}
% \end{macro}
%
% \begin{macro}{institute}
%   Set the institute on the title page.
%    \begin{macrocode}
\setbeamertemplate{institute}{
  \vspace*{3mm}
  \insertinstitute%
  \par%
}
%    \end{macrocode}
% \end{macro}
%
%
%
% \subsubsection{Section page}
%
% \begin{macro}{section page}
%
%   Template for the section title slide at the beginning of each section.
%
%    \begin{macrocode}
\newcommand{\@metropolis@sectionpage@none}{
  \AtBeginSection{
    % intenionally empty
  }
}
\defbeamertemplate{section page}{simple}{
  \centering
  \usebeamercolor[fg]{section title}
  \usebeamerfont{section title}
  \insertsectionhead\\
}
\newcommand{\@metropolis@sectionpage@simple}{
  \setbeamertemplate{section page}[simple]
  \AtBeginSection{
    \ifbeamer@inframe
      \sectionpage
    \else
      \frame[plain,c,noframenumbering]{\sectionpage}
    \fi
  }
}
\defbeamertemplate{section page}{progressbar}{
  \centering
  \begin{minipage}{22em}
    \usebeamercolor[fg]{section title}
    \usebeamerfont{section title}
    \insertsectionhead\\[-1ex]
    \usebeamertemplate*{progress bar in section page}
  \end{minipage}
  \par
}
\newcommand{\@metropolis@sectionpage@progressbar}{
  \setbeamertemplate{section page}[progressbar]
  \AtBeginSection{
    \ifbeamer@inframe
      \sectionpage
    \else
      \frame[plain,c,noframenumbering]{\sectionpage}
    \fi
  }
}
%    \end{macrocode}
% \end{macro}
%
% \begin{macro}{progress bar in section page}
%
%   Template for the progress bar displayed by default on the section page.
%   This code is duplicated in large part in the outer theme's template
%   |progress bar in head/foot|.
%
%    \begin{macrocode}
\newlength{\metropolis@progressonsectionpage}
\setbeamertemplate{progress bar in section page}{
  \setlength{\metropolis@progressonsectionpage}{%
    \textwidth * \ratio{\insertframenumber pt}{\inserttotalframenumber pt}%
  }%
  \begin{tikzpicture}
    \draw[bg, fill=bg] (0,0) rectangle (\textwidth, 0.4pt);
    \draw[fg, fill=fg] (0,0) rectangle (\metropolis@progressonsectionpage, 0.4pt);
  \end{tikzpicture}%
}
%    \end{macrocode}
%
%   The above code assumes that |\insertframenumber| is less than or equal to
%   |\inserttotalframenumber|. However, this is not true on the first compile;
%   in the absence of an |.aux| file, |\inserttotalframenumber| defaults to 1.
%   This behaviour could cause fatal errors for long presentations, as
%   |\metropolis@progressonsectionpage| would exceed \TeX's maximum length
%   (16383.99999pt, roughly 5.75 metres or 18.9 feet).
%   To avoid this, we increase the default value for |\inserttotalframenumber|;
%   presentations with over 4000 slides will still break on first compile, but
%   users in that situation likely have deeper problems to solve.
%
%    \begin{macrocode}
\def\inserttotalframenumber{100}
%    \end{macrocode}
% \end{macro}
%
%
%
% \subsubsection{Block environments}
%
% Regular block environment
%
%    \begin{macrocode}
\newlength{\@metropolis@blockskip}
\setbeamertemplate{block begin}{%
  \setlength{\parskip}{\@metropolis@parskip}
  \vspace*{1ex}
  \begin{beamercolorbox}[%
    ht=2.4ex,
    dp=1ex,
    leftskip=\@metropolis@blockskip,
    rightskip=\@metropolis@blockskip]{block title}
      \usebeamerfont*{block title}\insertblocktitle%
  \end{beamercolorbox}%
  \vspace*{-1pt}
  \usebeamerfont{block body}%
  \begin{beamercolorbox}[%
    dp=1ex,
    leftskip=\@metropolis@blockskip,
    rightskip=\@metropolis@blockskip,
    vmode]{block body}%
}
\setbeamertemplate{block end}{%
  \end{beamercolorbox}
  \vspace*{0.2ex}
}
%    \end{macrocode}
%
% Alerted block environment
%
%    \begin{macrocode}
\setbeamertemplate{block alerted begin}{%
  \setlength{\parskip}{\@metropolis@parskip}
  \vspace*{1ex}
  \begin{beamercolorbox}[%
    ht=2.4ex,
    dp=1ex,
    leftskip=\@metropolis@blockskip,
    rightskip=\@metropolis@blockskip]{block title alerted}
      \usebeamerfont*{block title alerted}\insertblocktitle%
  \end{beamercolorbox}%
  \vspace*{-1pt}
  \usebeamerfont{block body alerted}%
  \begin{beamercolorbox}[%
    dp=1ex,
    leftskip=\@metropolis@blockskip,
    rightskip=\@metropolis@blockskip,
    vmode]{block body alerted}%
}
\setbeamertemplate{block alerted end}{%
  \end{beamercolorbox}
  \vspace*{0.2ex}
}
%    \end{macrocode}
%
% Example block environment
%
%    \begin{macrocode}
\setbeamertemplate{block example begin}{%
  \setlength{\parskip}{\@metropolis@parskip}
  \vspace*{1ex}
  \begin{beamercolorbox}[%
    ht=2.4ex,
    dp=1ex,
    leftskip=\@metropolis@blockskip,
    rightskip=\@metropolis@blockskip]{block title example}
      \usebeamerfont*{block title example}\insertblocktitle%
  \end{beamercolorbox}%
  \vspace*{-1pt}
  \usebeamerfont{block body example}%
  \begin{beamercolorbox}[%
    dp=1ex,
    leftskip=\@metropolis@blockskip,
    rightskip=\@metropolis@blockskip,
    vmode]{block body example}%
}
\setbeamertemplate{block example end}{%
  \end{beamercolorbox}
  \vspace*{0.2ex}
}
%    \end{macrocode}
%
%
%
% \subsubsection{Lists and floats}
%
%    \begin{macrocode}
\setbeamertemplate{itemize items}{\textbullet}
\setbeamertemplate{caption label separator}{: }
\setbeamertemplate{caption}[numbered]
%    \end{macrocode}
%
%
%
% \subsubsection{Footnotes}
%    \begin{macrocode}
\setbeamertemplate{footnote}{%
  \parindent 0em\noindent%
  \raggedright
  \usebeamercolor{footnote}\hbox to 0.8em{\hfil\insertfootnotemark}\insertfootnotetext\par%
}
%    \end{macrocode}
%
%
%
% \subsubsection{Text and spacing settings}
%
%    \begin{macrocode}
\newlength{\@metropolis@parskip}
\setlength{\@metropolis@parskip}{0.5em}
\setlength{\parskip}{\@metropolis@parskip}
\linespread{1.15}
%    \end{macrocode}
%
% By default, Beamer frames offer the |c| option to \textit{almost} vertically
% center the text, but the placement is a little too high. To fix this, we
% redefine the |c| option to equalize |\beamer@frametopskip| and
% |\beamer@framebottomskip|. This solution was suggested by Enrico Gregorio in
% an answer to \href{http://tex.stackexchange.com/questions/247826/}{this
% Stack Exchange question}.
%
%    \begin{macrocode}
\define@key{beamerframe}{c}[true]{% centered
  \beamer@frametopskip=0pt plus 1fill\relax%
  \beamer@framebottomskip=0pt plus 1fill\relax%
  \beamer@frametopskipautobreak=0pt plus .4\paperheight\relax%
  \beamer@framebottomskipautobreak=0pt plus .6\paperheight\relax%
  \def\beamer@initfirstlineunskip{}%
}
%    \end{macrocode}
%
% Process package options
%
%    \begin{macrocode}
\@metropolis@inner@setdefaults
\ProcessPgfPackageOptions{/metropolis/inner}
%    \end{macrocode}
%
% \iffalse
%</package>
% \fi
% \Finale
\endinput
