%% ---------------------------------------------------------------------------
%% Copyright 2015 Matthias Vogelgesang and the LaTeX community. A full list of
%% contributors can be found at
%%
%%     https://github.com/matze/mtheme/graphs/contributors
%%
%% and the original template was based on the HSRM theme by Benjamin Weiss.
%%
%% This work is licensed under a Creative Commons Attribution-ShareAlike 4.0
%% International License (https://creativecommons.org/licenses/by-sa/4.0/).
%% ---------------------------------------------------------------------------

\documentclass{ltxdoc}
%\OnlyDescription

\usepackage{parskip}
\usepackage{setspace}
\onehalfspacing

\usepackage{fontspec}
\setmainfont[BoldItalicFont={Fira Sans Italic},%
             ItalicFont={Fira Sans Light Italic},%
             BoldFont={Fira Sans}]{Fira Sans Light}
\setmonofont{Fira Mono}
\defaultfontfeatures{Ligatures=TeX}

\usepackage{enumitem}
\setlist[itemize]{noitemsep}
\setlist[enumerate]{noitemsep}

\usepackage{xcolor}
\definecolor{mDarkBrown}{HTML}{604c38}
\definecolor{mDarkTeal}{HTML}{23373b}
\definecolor{mLightBrown}{HTML}{EB811B}
\definecolor{mLightGreen}{HTML}{14B03D}
\definecolor{mBackground}{HTML}{FFFFFF}

\usepackage{listings}
\lstset{%
  language=[LaTeX]{TeX},
  basicstyle=\ttfamily,
  keywordstyle=\color{mLightBrown}\bfseries,
  commentstyle=\color{mLightGreen},
  stringstyle=\color{mLightGreen},
  backgroundcolor=\color{mBackground},
  numbers=none,
  numberstyle=\tiny\ttfamily,
  stepnumber=2,
  showspaces=false,
  showstringspaces=false,
  showtabs=false,
  frame=none,
  framerule=1pt,
  tabsize=2,
  rulesep=5em,
  captionpos=b,
  breaklines=true,
  breakatwhitespace=false,
  framexleftmargin=0em,
  framexrightmargin=0em,
  xleftmargin=0em,
  xrightmargin=0em,
  aboveskip=1em,
  belowskip=1em,
  morekeywords={usetheme,institute,maketitle,@metropolis@titleformat,plain,setbeamercolor,metropolisset},
}
\lstMakeShortInline|

\usepackage[colorlinks=true,
            linkcolor=mLightBrown,
            menucolor=mLightBrown,
            pagecolor=mLightBrown,
            urlcolor=mLightBrown]{hyperref}

\newcommand{\DescribeOption}[4]{
  \DescribeMacro{#1}
  \begin{minipage}[t]{\textwidth}
    \textit{\textbf{\textcolor{mLightGreen}{#2}}}\dotfill\,#3\par
    \begingroup
    \leftskip2em #4\par
    \endgroup
  \end{minipage}
}

\GetFileInfo{beamerthemem.dtx}
\title{Modern Beamer Presentations\\with the\\\textsc{mtheme} package}
\author{Matthias Vogelgesang \\ \url{matthias.vogelgesang@gmail.com}}
\date{v0.x.x}

\begin{document}

\maketitle

\section{Introduction}
Beamer is an awesome way to make presentations with LaTeX. But the stock
themes do not necessarily look particularly nice and the custom themes often
scream “Beamer” at first sight. The goal of \textsc{mtheme} is to provide a
modern Beamer theme with minimal visual noise. It provides section slides with
a neat progress bar and it is intended to be used with
\href{https://www.mozilla.org/en-US/styleguide/products/firefox-os/typeface/}
{Fira Sans}, a gorgeous typeface commissioned by Mozilla and designed by
\href{http://www.carrois.com/fira-3-1/}{Carrois}. Hence to get the best results
you should have installed the Fira typeface and use XeTeX to typeset your
slides. Nevertheless this is no hard dependency. The theme also works fine with
pdfTeX and the Computer Modern typeface.

The codebase is maintained on \href{https://github.com/matze/mtheme}{GitHub}.
So if you have issues, find mistakes in the manual or want to contribute – to
make the theme even better – get in touch there.


\section{Getting Started}

\subsection{Installation}

The \textsc{mtheme} uses Make as build system. Hence the installation is very
straight forward. Simply type
\begin{lstlisting}
$ make
$ make install
\end{lstlisting}
in the top directory and all the files will be created and installed on your
computer. The complete list of make rules is as follows:

\begin{description}
  \item[all] \hfill \\
  Build the theme, the manual and the demo presentation.
  \item[install] \hfill \\
  Install the theme into your local texmf folder.
  \item[uninstall] \hfill \\
  Remove the theme from your local texmf folder.
  \item[sty] \hfill \\
  Creat the package files.
  \item[doc] \hfill \\
  Build the documentation.
  \item[demo] \hfill \\
  Build the demo presentation.
  \item[demo-min] \hfill \\
  Build the minimal demo presentation.
  \item[ctan] \hfill \\
  Create a package for CTAN distribution.
\end{description}


\subsection{Dependencies}

\begin{itemize}
  \item XeLaTeX
  \item \href{https://github.com/mozilla/Fira}{Fira Sans} and Mono font
  \item TikZ
\end{itemize}

Depending on the Linux distribution, the packaged name of |Fira Sans| might be
|Fira Sans OT| instead of |Fira Sans|. In that case, you may have to edit
|beamerfontthememetropolis.dtx|. You may also need to install Fira Sans; see
the |contrib/| directory for more. Users of Debian or Ubuntu can also install
this \href{https://launchpad.net/\%7Eedd/+archive/ubuntu/misc/+files/latex-mtheme_0.1.0vidid1_all.deb}{.deb package} containing the theme files as well
as the Fira Sans font files.


\subsection{Pandoc}

To use this theme with \href{http://johnmacfarlane.net/pandoc/}{Pandoc}-based
presentations, you can run the following command

\begin{lstlisting}
$ pandoc -t beamer --latex-engine=xelatex -V theme:m -o output.pdf input.md
\end{lstlisting}


\subsection{A Minimal Example}

To get started with the theme is very simple. The following code shows a
minimal example of a Beamer presentation using the \textsc{mtheme}.

\begin{lstlisting}
\documentclass[10pt]{beamer}
\usetheme{m}                          % load mtheme
\title{A modern beamer theme}         % define title
\date{\today}                         % define date
\author{Matthias Vogelgesang}         % define author
\institute{Institute}                 % define institute
\begin{document}
\maketitle                            % create titlepage
\section{First Section}               % create section
\begin{frame}{First Frame}            % first frame
  Lorem ipsum dolor sit amet, ...
\end{frame}
\begin{frame}{Second Frame}           % second frame
  Lorem ipsum dolor sit amet, ...
\end{frame}
\end{document}
\end{lstlisting}

\section{Customization}
\subsection{Package options}
The theme provides a number of options. The options use a key=value interface.
So every option is controlled by a key its value. To use an option you can
either provide a comma separated list of options when invoking
\textsc{mtheme} in the preamble of the presentation.
\begin{lstlisting}
\usetheme[<key=value list>]{m}
\end{lstlisting}
Or you can set them at any time with the |\metropolisset| macro.
\begin{lstlisting}
\metropolisset[<key=value list>]
\end{lstlisting}
To set an option on a specific sub-package only you have to add the
corresponding prefix (inner, outer, color), e.g.
\begin{lstlisting}
\metropolisset[inner/block=fill]
\end{lstlisting}
The list of options is structured as shown in the following example.

\DescribeOption{key}{list of possible values}{default value}{
  A short description of the option.
}

Although the options are grouped into the corresponding packages every option
can and in most cases should be set on the main theme directly. If an option
is listed in multiple sub-packages, setting it on the main theme will set the
option on every sub-package accordingly.

\subsubsection{Main theme}
\DescribeOption{everytitleformat}{regular, lowercase, uppercase}{lowercase}{
  Shortcut option to change the case style of all titles together.
}

\DescribeOption{plainformat}{regular, lowercase, uppercase}{lowercase}{
  Control the case style of the plain title.
}

\subsubsection{Inner theme}
\DescribeOption{block}{transparent, fill}{transparent}{
  This option controls the block background. It can either be filled with a
  light grey or be transparent.
}

\DescribeOption{sectionpage}{none, progressbar}{progressbar}{
  Adds a thin progress bar similar to the section progress bar underneath each frame title.
}

\DescribeOption{titleformat}{regular, lowercase, uppercase}{lowercase}{
  Control the case style of the title.
}

\DescribeOption{sectiontitleformat}{regular, lowercase, uppercase}{lowercase}{
  Control the case style of the section title.
}

\subsubsection{Outer theme}
\DescribeOption{numbering}{none, counter, fraction}{counter}{
  In the bottom right corner of each frame the current frame number is
  displayed. This can be disabled or the total framenumber can be added
  additionally.
}

\DescribeOption{progressbar}{none, head}{none}{
  If enabled this option adds a thin progress bar similar to the sections
  progress bar underneath each frame title.
}

\DescribeOption{frametitleformat}{regular, lowercase, uppercase}{lowercase}{
  Control the case style of the frame title.
}

\DescribeOption{frametitleoffset\\noframetitleoffset}{<dimension>}{2em}{
  The frametitle offset is an additional vertical space after the frame title
  to center the content vertically on the frame. To remove this space entirely
  the short option |noframetitleoffset| is defined.
}

\subsubsection{Color theme}
\DescribeOption{block}{transparent, fill}{transparent}{
  This option controls the block background. It can either be filled with a
  light grey or be transparent.
}

\DescribeOption{background}{dark, light}{light}{
  This option defines whether the background shall be dark and the foreground
  be light or vice versa.
}

\subsection{Color Customization}

The included \textsc{metropolis} color theme is used by default, but its
colors can be easily changed to suit your tastes. All of the theme's styles
are defined in terms of three beamer colors:
\begin{itemize}
    \item |normal text| (dark fg, light bg)
    \item |alerted text| (colored fg, should be visible against dark or light)
    \item |example text| (colored fg, should be visible against dark or light)
\end{itemize}
An easy way to customize the theme is to redefine these colors using

\begin{lstlisting}
\setbeamercolor{ ... }{ fg= ... , bg= ... }
\end{lstlisting}
in your preamble. For greater customization, you can redefine any of the other
stock beamer colors. In addition to the stock colors the theme defines a number
of \textsc{metropolis} specific colors, which can also be redefined to your
liking.

\begin{lstlisting}
\setbeamercolor{progress bar}{ ... }
\setbeamercolor{title separator}{ ... }
\setbeamercolor{progress bar in head/foot}{ ... }
\setbeamercolor{progress bar in section page}{ ... }
\end{lstlisting}

\subsection{Commands}

The |\plain{title=[]}{<body>}| command sets a slide in plain dark colors which
can be useful to focus attention on a single sentence or image.

\subsection{Paul Tol's colors: a \texttt{pgfplots} theme}

A good presentation uses colors that are

\begin{itemize}
  \item distinct from each other as much as possible, and
  \item distinct from black and white,
  \item under many different lighting and display environments, and
  \item to color-blind viewers,
  \item all while matching well together.
\end{itemize}

In a \href{https://personal.sron.nl/~pault/colourschemes.pdf}{technical note}
for SRON, Paul Tol proposed a palette of colors satisfying these constraints.
The sub-package |pgfplotsthemetol| defines palettes for |pgfplots| charts
based on Tol's work. Use the |mlineplot| key to plot line data and |mbarplot|
or horizontal |mbarplot| to plot bar charts.


\section{Known Issues}


\section{License}

The theme itself is licensed under a
\href{http://creativecommons.org/licenses/by-sa/4.0/}{Creative Commons
Attribution-ShareAlike 4.0 International License}. This means that if you change
the theme and re-distribute it, you must retain the copyright notice header and
license it under the same CC-BY-SA license. This does not affect the
presentation that you create with the theme.


\section{Contributors}

For a full list of contributors please visit the
\href{https://github.com/matze/mtheme/graphs/contributors}{GitHub Repository}.


\section{Implementation}

\DocInput{beamerthemem.dtx}
\DocInput{beamerinnerthememetropolis.dtx}
\DocInput{beamerouterthememetropolis.dtx}
\DocInput{beamerfontthememetropolis.dtx}
\DocInput{beamercolorthememetropolis.dtx}
\DocInput{pgfplotsthemetol.dtx}

\end{document}
