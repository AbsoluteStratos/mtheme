%% ---------------------------------------------------------------------------
%% Copyright 2015 Matthias Vogelgesang and the LaTeX community. A full list of
%% contributors can be found at
%%
%%     https://github.com/matze/mtheme/graphs/contributors
%%
%% and the original template was based on the HSRM theme by Benjamin Weiss.
%%
%% This work is licensed under a Creative Commons Attribution-ShareAlike 4.0
%% International License (https://creativecommons.org/licenses/by-sa/4.0/).
%% ---------------------------------------------------------------------------

\documentclass{ltxdoc}
%\OnlyDescription

\usepackage[parfill]{parskip}
\usepackage{setspace}
\onehalfspacing

\usepackage{fontspec}
\setmainfont[BoldItalicFont={Fira Sans Italic},%
             ItalicFont={Fira Sans Light Italic},%
             BoldFont={Fira Sans}]{Fira Sans Light}
\setmonofont{Fira Mono}
\defaultfontfeatures{Ligatures=TeX}

\usepackage{enumitem}
\setlist[itemize]{noitemsep}
\setlist[enumerate]{noitemsep}

\usepackage{xcolor}
\definecolor{mDarkBrown}{HTML}{604c38}
\definecolor{mDarkTeal}{HTML}{23373b}
\definecolor{mLightBrown}{HTML}{EB811B}
\definecolor{mLightGreen}{HTML}{14B03D}
\definecolor{mBackground}{HTML}{FFFFFF}

\usepackage{listings}
\lstset{%
  language=[LaTeX]{TeX},
  basicstyle=\ttfamily,
  keywordstyle=\color{mLightBrown}\bfseries,
  commentstyle=\color{mLightGreen},
  stringstyle=\color{mLightGreen},
  backgroundcolor=\color{mBackground},
  numbers=none,
  numberstyle=\tiny\ttfamily,
  stepnumber=2,
  showspaces=false,
  showstringspaces=false,
  showtabs=false,
  frame=none,
  framerule=1pt,
  tabsize=2,
  rulesep=5em,
  captionpos=b,
  breaklines=true,
  breakatwhitespace=false,
  framexleftmargin=0em,
  framexrightmargin=0em,
  xleftmargin=0em,
  xrightmargin=0em,
  aboveskip=1em,
  belowskip=1em,
  morekeywords={usetheme,institute,maketitle,mthemetitleformat,plain,setbeamercolor},
}
\lstMakeShortInline|

\usepackage[colorlinks=true,
            linkcolor=mLightBrown,
            menucolor=mLightBrown,
            pagecolor=mLightBrown,
            urlcolor=mLightBrown]{hyperref}

\GetFileInfo{beamerthemem.dtx}
\title{Modern Beamer Presentations\\with the\\\textsc{mtheme} package}
\author{Matthias Vogelgesang \\ \url{matthias.vogelgesang@gmail.com}}
\date{v0.x.x}

\begin{document}

\maketitle

\section{Introduction}
Beamer is an awesome way to make presentations with LaTeX. But the stock themes
do not necessarily look particularly nice and the custom themes often scream
“Beamer” at first sight. The goal of \textsc{mtheme} is to provide a modern
Beamer theme with minimal visual noise. It provides section slides with a neat
progress bar and It is intended to be used with
\href{https://www.mozilla.org/en-US/styleguide/products/firefox-os/typeface/}{Fira
Sans}, a gorgeous typeface commissioned by Mozilla and designed by
\href{http://www.carrois.com/fira-3-1/}{Carrois}. Hence to get the best results
you should have installed the Fira typeface and use XeTeX to typeset your
slides. Nevertheless this is no hard dependency. The theme also works fine with
pdfTeX and the Computer Modern typeface.

The codebase is maintained on \href{https://github.com/matze/mtheme}{GitHub}. So
if you have issues, find mistakes in the manual or want to contribute – to make
the theme even better – get in touch there.


\section{Getting Started}

\subsection{Installation}

The \textsc{mtheme} uses Make as build system. Hence the installation is very
straight forward. Simply type
\begin{lstlisting}
$ make
$ make install
\end{lstlisting}
in the top directory and all the files will be created and installed on your
computer. The complete list of make rules is as follows:

\begin{description}
  \item[all] \hfill \\
  Build the theme, the manual and the demo presentation.
  \item[install] \hfill \\
  Install the theme into your local texmf folder.
  \item[uninstall] \hfill \\
  Remove the theme from your local texmf folder.
  \item[sty] \hfill \\
  Build the manual.
  \item[manual] \hfill \\
  Build the manual.
  \item[demo] \hfill \\
  Build the demo presentation.
  \item[ctan] \hfill \\
  Create a package for CTAN distribution.
\end{description}


\subsection{Dependencies}

\begin{itemize}
  \item XeLaTeX
  \item \href{https://github.com/mozilla/Fira}{Fira Sans} and Mono font
  \item TikZ
\end{itemize}

Depending on the Linux distribution, the packaged name of |Fira Sans| might be
|Fira Sans OT| instead of |Fira Sans|. In that case, you may have to edit
|beamerfontthememetropolis.dtx|. You may also need to install Fira Sans; see
the |contrib/| directory for more. Users of Debian or Ubuntu can also install
this \href{https://launchpad.net/\%7Eedd/+archive/ubuntu/misc/+files/latex-mtheme_0.1.0vidid1_all.deb}{.deb package} containing the theme files as well
as the Fira Sans font files.


\subsection{Pandoc}

To use this theme with \href{http://johnmacfarlane.net/pandoc/}{Pandoc}-based
presentations, you can run the following command

\begin{lstlisting}
$ pandoc -t beamer --latex-engine=xelatex -V theme:m -o output.pdf input.md
\end{lstlisting}


\subsection{A Minimal Example}

To get started with the theme is very simple. The following code shows a
minimal example of a Beamer presentation using the \textsc{mtheme}.

\begin{lstlisting}
\documentclass[10pt]{beamer}
\usetheme{m}                          % load mtheme
\title{A modern beamer theme}         % define title
\date{\today}                         % define date
\author{Matthias Vogelgesang}         % define author
\institute{Institute}                 % define institute
\begin{document}
\maketitle                            % create titlepage
\section{First Section}               % create section
\begin{frame}{First Frame}            % first frame
  Lorem ipsum dolor sit amet, ...
\end{frame}
\begin{frame}{Second Frame}           % second frame
  Lorem ipsum dolor sit amet, ...
\end{frame}
\end{document}
\end{lstlisting}

\section{Customization}
\subsection{Package options}
The theme provides a number of options. To use any of the options below, call
them when invoking \textsc{mtheme} in the preamble of the slides, i.e.
\begin{lstlisting}[numbers=none]
\usetheme[<options>]{m}
\end{lstlisting}

\noindent\DescribeMacro{usetitleprogressbar}
Adds a thin progress bar similar to the section progress bar underneath each
frame title.\\

\noindent\DescribeMacro{blockbg}
Adds background color to the blocks similar to other beamer themes.\\

\noindent\DescribeMacro{nooffset}
By default, the \textsc{mtheme} adds |\vspace{2em}| after the frametitle to
center content vertically on the frame. This option removes this additional
space in order to get more content per slide.\\

\noindent\DescribeMacro{nosectionslide}
By default when using the |\section| command, a slide is created with just the
title and the progress bar on it. This option prevents the creation of these
additional slides.\\

\noindent\DescribeMacro{usetotalslideindicator}
By default, only the current page number is printed in the lower right corner.
This option changes the slide numbering format to |#current/#total|.\\

\noindent\DescribeMacro{noslidenumbers}
Omits slide numbers entirely.\\

\noindent\DescribeMacro{darkcolors}
Makes the background dark and the foreground light.\\


\subsection{Color Customization}

The included metropolis color theme is used by default, but its colors can be
easily changed to suit your tastes. All of the theme's styles are defined in
terms of three beamer colors:
\begin{itemize}
    \item |normal text| (dark fg, light bg)
    \item |alerted text| (colored fg, should be visible against dark or light)
    \item |example text| (colored fg, should be visible against dark or light)
\end{itemize}
An easy way to customize the theme is to redefine these colors using

\begin{lstlisting}[numbers=none]
\setbeamercolor{ ... }{ fg= ... , bg= ... }
\end{lstlisting}
in your preamble. For greater customization, you can redefine any of the other
colors in |beamercolorthememetropolis|, including progress bar.


\subsection{Title Case Formatting}

The main title, section titles, frame titles and plain frame titles are all
formatted according to the custom command |\mthemetitleformat|. By default,
this is equivalent to |\MakeLowercase{#1}|, hence seting the titles in small
capitals. You can change this behaviour in your preamble. For example:

\begin{lstlisting}
% camel case
\renewcommand{\mthemetitleformat}{}
% lowercase
\renewcommand{\mthemetitleformat}\MakeLowercase}
% uppercase
\renewcommand{\mthemetitleformat}{\MakeUppercase}
\end{lstlisting}

Be aware that these formatting macros will be replaced with theme options in
the future.

\subsection{Commands}

The |\plain{title=[]}{<body>}| command sets a slide in plain dark colors which
can be useful to focus attention on a single sentence or image.


\subsection{Paul Tol's colors: a \texttt{pgfplots} theme}

A good presentation uses colors that are

\begin{itemize}
  \item distinct from each other as much as possible, and
  \item distinct from black and white,
  \item under many different lighting and display environments, and
  \item to color-blind viewers,
  \item all while matching well together.
\end{itemize}

In a \href{https://personal.sron.nl/~pault/colourschemes.pdf}{technical note}
for SRON, Paul Tol proposed a palette of colors satisfying these constraints.
The sub-package |pgfplotsthemetol| defines palettes for |pgfplots| charts
based on Tol's work. Use the |mlineplot| key to plot line data and |mbarplot|
or horizontal |mbarplot| to plot bar charts.


\section{Known Issues}


\section{License}

The theme itself is licensed under a
\href{http://creativecommons.org/licenses/by-sa/4.0/}{Creative Commons
Attribution-ShareAlike 4.0 International License}. This means that if you change
the theme and re-distribute it, you must retain the copyright notice header and
license it under the same CC-BY-SA license. This does not affect the
presentation that you create with the theme.


\section{Contributors}

For a full list of contributors please visit the
\href{https://github.com/matze/mtheme/graphs/contributors}{GitHub Repository}.


\section{Implementation}

\DocInput{beamerthemem.dtx}
\DocInput{beamerinnerthememetropolis.dtx}
\DocInput{beamerouterthememetropolis.dtx}
\DocInput{beamerfontthememetropolis.dtx}
\DocInput{beamercolorthememetropolis.dtx}
\DocInput{pgfplotsthemetol.dtx}

\end{document}
